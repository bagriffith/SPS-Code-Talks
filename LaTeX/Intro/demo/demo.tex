\documentclass{article}
\usepackage{amsmath}
\usepackage[american]{circuitikz}
\usepackage{fancyhdr}
\usepackage{graphicx}
\usepackage{listings}

\pagestyle{fancy}
\fancyhead{}
\fancyfoot{}
\fancyhead[L]{Brady Griffith}
\fancyhead[C]{\LaTeX \ Example}
\fancyhead[R]{\thepage}

\title{\LaTeX \ Example}
\author{Brady Griffith}
\date{October 03, 2018}

\begin{document}

\maketitle
\newpage

\section*{Question 1}
\subsection*{Part A}
The velocity of the particle is given by $v(t) = at$.

To find the position, velocity is integrated with respect to time.

$$ x(t) = \int_{t_0}^{t_1} v(t) dt = \frac{a}{2} t^2 + x_0 $$

\subsection*{Part B}
The positions for 1s and 2s are given. \\
\begin{tabular}{ l | l }
  t (s) & x (m) \\ \hline
  1 & $\frac{1}{2}$ \\
  2 & 2 \\
\end{tabular}

\section*{Question 2}
The vector field $\vec{v} = \left< y, -x \right>$ is shown.

\includegraphics[width=2in]{example}

\section*{Question 3}
The following equations
$$ 2x + 3y $$
$$ 4x + 5y $$
are given by the following matrix
\[
 \begin{bmatrix}
   2 & 3 \\
   4 & 5
 \end{bmatrix}
\]

\section*{Question 4}
\begin{align*}
  a &= 2x + 2y \\
  &= 2(x + y)
\end{align*}
  
\section*{Question 5}
\begin{circuitikz} \draw
  (0,0) to[short, o-] (-6,0)
        to[american voltage source,l=$V$] (-6,4) -- (-4,4)
        to[short, -o] (0,4)
        (-2,0) to[R, l=$R_1$] (-2,4)
        ;
\end{circuitikz}

\section*{Question 6}
\begin{lstlisting}[language=FORTRAN, caption=FORTRAN example]
  PROGRAM EXAMPLE
    INTEGER :: N = 3 
    N = N + 1
    PRINT *, N
  END PROGRAM EXAMPLE
\end{lstlisting}

\end{document}
